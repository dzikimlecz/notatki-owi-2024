\documentclass{article}
\usepackage{graphicx}
\graphicspath{ {./grafiki/} }
\usepackage{transparent}
\usepackage[T1]{fontenc}
\usepackage[a4paper, total={15cm, 24cm}]{geometry}
\usepackage[polish]{babel}
\usepackage{amssymb}
\usepackage{amsmath}
\usepackage[
  pdfpagelabels=true,
  pdftitle={Notatki z OWI},
  pdfauthor={Jan Pulkowski},
  colorlinks=true,
  linkcolor=blue,
  ]{hyperref}
\usepackage[]{titlesec}
\usepackage[p]{scholax}

\newcommand{\watermark}{
  \begin{figure}[b]
    \transparent{0.4} \includegraphics*[]{mlecz}
    \centering
  \end{figure}
}

\setcounter{secnumdepth}{2}

\renewcommand{\thesection}{\arabic{section}}

\titleformat{\section}{\normalfont\Large\bfseries\scshape}{Wykład \thesection:}{1em}{}

\title{Notatki z Ochrony Własności Intelektualnej}
\author{mlecz}
\date{Semestr Letni 2024}

\begin{document}

\maketitle
\tableofcontents
\watermark

\newpage
\section{Prawo i jego źródła (28.02)}

\subsection{Prawo}

\paragraph{Prawo w ujęciu normatywnym}
to zespół norm i reguł określających postępowanie ludzi, ustanowionych, usankcjonowanych i zabezpieczanych przez aparat przymusu państwowego.

\paragraph{Prawo wg Kanta}
to reguły pozwalające na pogodzenie samowoli dwóch niezależnych jednostek.

\subsubsection{Zasady Prawa}
\begin{itemize}
  \item Lex posterior derogat legi priori -- prawo późniejsze uchyla prawo wcześniejsze
  \item Lex superiori derogat legi inferiori -- prawo o wyższej mocy uchyla prawo o niższej mocy
  \item Lex specialis derogat legi generali -- prawo szczegółowe uchyla prawo ogólne
  \item Ne bis in idem -- Nie można orzekać dwa razy w tej samej sprawie
  \item Lex retro non agit -- Prawo nie działa wstecz (o ile ustawa późniejsza nie jest
        korzystniejsza dla oskarżonego)
\end{itemize}

\subsubsection{Miejsca zapisu prawa}
\begin{itemize}
  \item \textbf{Dziennik ustaw -- jedyne oficjalne miejsce zapisu prawa}
  \item Monitor Polski -- służy do ogłaszania wewnętrznych aktów prawnych wydawanych
        przez organy państwowe
  \item ISAP -- Internetowy System Aktów Prawnych
  \item Systemy Prawne (LEX, Legalis) -- ujednolicony i czytelny zapis informacji
        z dziennika ustaw z odnośnikami do innych powiązanych informacji prawnych
\end{itemize}

\subsection{Język prawny a prawniczy}

\paragraph{Język prawny} to język tworzenia prawa, w którym tworzone są akty prawne.
Nie musi być poprawny w sensie składni j. pol. ale jest mniej wieloznaczny i ściślejszy
od języka naturalnego.

\paragraph{Język prawniczy} to język stosowania i doktryny prawa, omawiający treść
i interpretację ustawy.

\paragraph{Język prawny a matematyczny}
Język prawny często celowo zostawia pewne pojęcia niedookreślone,aby pozostawić
sądom pole do odmiennej wykładni prawa w różnych przypadkach.

\subsection{Obowiązywanie Prawa}

\paragraph{Prawo powszechnie obowiązujące} to przepisy adresowane do wszystkich podmiotów.
Wyznacza ich sytuację prawną, tzn. prawa i obowiązki.
Stanowi fundament działania państwa,
jest umową społeczną między jego państwa i obywatelami.
Państwo może wykonywać jedynie czynności \textit{explicite} dozwolone przez prawo,
a obywatele wszystkie czynności niezakazane.

\subsection{Źródła prawa}
\begin{enumerate}
  \item Konstytucja
  \item Ustawy
  \item Ratyfikowane umowy międzynarodowe
  \item Rozporządzenia
  \item Akty prawa miejscowego (tylko na obszarze obowiązywania)
\end{enumerate}

\paragraph{Konstytucja}-- akt prawny nadrzędny wobec wszystkich pozostałych.
Jej przepisy stosuje się bezpośrednio, o ile sama nie stanowi inaczej.
Jest jednym, spójnym aktem prawnym. Konstytucja jest celowo napisana w sposób ogólny, aby mogły
uszczegółowić ją ustawy.
Określa bazowe ramy prawne, sposób stanowienia prawa i zasady jakie można z niej wyprowadzać.

\paragraph{Ustawa, a rozporządzenie} -- Ustawa to prawo stanowione przez władzę ustawodawczą,
rozporządzenie jest aktem wykonawczym, doprecyzowuje działanie i stosowanie ustawy oraz określa
szczegółowe procedury jej realizacji. Rozporządzenia są tworzone na podstawie ustaw.

\subsection{Prawo unijne}

\paragraph{Prawo pierwotne}
to umowy międzynarodowe tworzące i kształtujące Unię Europejską. Stanowią jej swoistą konstytucję.
Na przykład Traktat z Lizbony powołujący Unię.

\paragraph{Prawo wtórne}
to rozporządzenia, dyrektywy, decyzje, opinie i zalecenia wydawane przez organy Unii.

\paragraph{Rozporządzenia unijne} to akty prawne
bezpośrednio i jednolicie działające we wszystkich krajach członkowskich.

\paragraph{Dyrektywy unijne} to akty prawne,
wskazujące cele i wytyczne, które ustawodawstwo poszczególnych krajów członkowskich powinno zrealizować.
Muszą zostać wydane krajowe akty prawne, które implementują dyrektywę.

\subsubsection{Prawo polskie a prawo unijne}
Konstytucja dopuszcza przekazanie części uprawnień organów państwowych organom prawa międzynarodowego.
Proeuropejska wykładnia prawa jest ponad krajową --
ratyfikowana umowa międzynarodowa ma pierwszeństwo wobec sprzecznej z nią ustawy.
Nad umowami międzynarodowymi stoi jedynie konstytucja.


\section{Własność Intelektualna (06.03)}

\subsection{Dobra Niematerialne}

\paragraph{Dobro niematerialne (intelektualne)}
to wytwór myśli, cechujący się twórczym charakterem, istniejący w świadomości człowieka
i mogący podlegać ochronie prawnej niezależnie od~swojego ewentualnego nośnika \textit{(Corpus Mechanicum)}.

\subsubsection{Rodzaje dóbr niematerialnych}
\begin{itemize}
  \item Utwory (w rozumieniu prawa autorskiego)
  \item Rozwiązania -- zaplanowane koncepty, prowadzące do realizacji danego celu (wynalazki, projekty, wzory)
  \item Oznaczenia, symbole (znaki towarowe, oznaczenia geograficzne)
\end{itemize}

Dobra niematerialne mogą obejmować więcej niż jeden z powyższych aspektów.

\subsection{Prawo własności intelektualnej}

\paragraph{Prawo autorskie i pokrewne}
zajmuje się utworami. Godzi prawa twórców z możliwością rozwoju społeczeństwa. Chroni prawa twórców jak i wykonawców. Regulowane międzynarodowo.

\subsection{Konwencje o ochronie prawa autorskiego}

\subsubsection{Konwencja Berneńska}

\paragraph{Konwencja Berneńska}
o ochronie dzieł literackich i artystycznych z 1886 roku. Powstała z~inicjatywy m. in. Wiktora Hugo.
Wprowadziła koncept międzynarodowej ochrony praw autora.
Ratyfikowana przez większość państw świata, wielokrotnie nowelizowana. Nadal obowiązuje.

\paragraph{Zasada wzajemnego respektowania praw} Utworom zagranicznym zapewnia się taką samą ochronę jak utworom krajowym.

\paragraph{Zasada automatyzmu}
Ochrona praw autorskich przysługuje każdemu utworowi od momentu jego powstania.

\subsubsection{Inne Konwencje}

\paragraph{Konwencja genewska} o prawach autorskich.
Zapewnia alternatywę wobec Konwencji Berneńskiej.
Reprezentuje bardzie liberalne podejście do sprawy.

\paragraph{Konwencja Rzymska}
o ochronie wykonawców, producentów fonogramów oraz organizacji nadawczych.
Zapewnia ochronę twórcom utworów dźwiękowych.

\paragraph{Porozumienie w sprawie Handlowych Aspektów Praw Własności Intelektualnej (TRIPS)}
poza zakresem Konwencji Berneńskiej zapewniło ochronę programów komputerowych i baz danych.
Chroni programy w postaci kodu źródłowego i przedmiotowego (postać binarna, wynikowa, ang. \textit{object files}).

\paragraph{Traktaty Światowej Organizacji Własności Intelektualnej (WIPO)}
o prawie autorskim, o artystycznych wykonaniach i fonogramach.
Pierwszy traktat międzynarodowy nakładający obowiązek chronienia programów komputerowych i baz danych.
Wymusza wprowadzenie sankcji za łamanie i obchodzenie zabezpieczeń programów.

\subsection{Ustawa o prawach autorskich i pokrewnych}

\subsubsection{Utwory}

\paragraph{Utwór} to każdy przejaw działalności twórczej o indywidualnym charakterze, ustalony w~jakiejkolwiek postaci,
niezależnie od wartości, przeznaczenia i sposobu wyrażenia.
Definicja celowo jest szeroka aby móc pozostawić kwestię, czy dana rzecz jest utworem do interpretacji sądowej.

\paragraph{Konkretne przypadki wykładni definicji utworu}
\begin{itemize}
  \item Smak i zapach nie może podlegać ochronie prawnoautorskiej. Subiektywne doznania odgrywają tu zbyt dużą rolę.
  \item By mówić o utworze konieczna jest cecha nowości, twórca musiał wprowadzić uchwytne elementy, nieobecne w dotychczasowym stanie rzeczy.
  \item Jeżeli wysoce prawdopodobne jest, że ktoś w przyszłości wytworzy \textbf{identyczny} przedmiot, to nie mówi się o utworze.
\end{itemize}

\paragraph{Zasada Ustalenia}
-- Utwór musi zostać przedstawiony, w postaci która pozwala na jego percepcję komuś innemu niż twórcy, nawet ulotnej, nietrwałej.
Utwór może istnieć bez nośnika, a posiadanie nośnika nie jest związane z prawem do utworu.

\section{Twórcy i ochrona utworów (13.03)}

\subsection{Osoba twórcy}

\paragraph{Twórcą}
utworu może być dowolna osoba, niezależnie od stanu świadomości.
W porządku prawnym amerykańskim i polskim przesądzono, że zwierzęta ani natura nie mogą być twórcą utworu.
Ogólnie, twórcą może być wyłącznie człowiek.
Często uważa się, że wytwory systemu niedeterministycznych (np. AI) należą do ich operatorów,
natomiast nie jest to uniwersalna wykładnia i silnie zależy od konkretnego przypadku.
Obecny porządek prawny niedostatecznie określa interpretacje praw do wytworów sztucznej inteligencji,
która nie może być uznana za~twórcę utworu.

\subsection{Ochrona utworów}

Utwór jest chroniony od momentu powstania, nawet w postaci nieukończonej.
Ochrona przysługuje mu niezależnie od spełnienia jakichkolwiek formalności.
Ochronie nie może podlegać procedura, koncepcja matematyczna ani odkrycia,
a jedynie sposób wyrażenia (nie temat a jego indywidualizacja).
Dyrektywy unijne i inne przepisy doprecyzowują kryteria uznania tworu konkretnego typu za utwór.

\paragraph{Czy każde dzieło jest chronione jako utwór?}
tldr.: Nie.
\begin{figure}[h]
  \begin{minipage}{0.31\textwidth}
    \includegraphics[width=0.95\textwidth]{malewiczkwadrat}
    \caption{K. Malewicz -- Czarny kwadrat na białym tle}\label{fig:malewiczkwadrat}
  \end{minipage}
  \hfill
  \begin{minipage}{0.31\textwidth}
    \includegraphics[width=0.95\textwidth]{behandomek}
    \vspace{-\baselineskip}
    \caption{A. Behan -- 2023}\label{fig:behandomek}
  \end{minipage}
  \hfill
  \begin{minipage}{0.31\textwidth}
    \includegraphics[width=0.95\textwidth]{iosononiewidzialne}
    \vspace{-\baselineskip}
     \caption{S. Garau -- Io Sono}\label{fig:iosononiewidzialne}
  \end{minipage}
\end{figure}

Z powyższych utworem jest tylko Rysunek 2.
Niewidzialna rzeźba (3) istnieje jedynie jako koncepcja, nie ma żadnej ustalonej formy -- nie jest utworem.
Natomiast czarny kwadrat (1) pomimo oryginalnej koncepcji, nie spełnia kryterium oryginalności.
Prawdopodobieństwo wytworzenia identycznego dzieła jest zbyt duże by mówić o utworze.

\subsection{Ochrona programów komputerowych}

\paragraph{Programy Komputerowe}
podlegają ochronie w każdej postaci (kod, kod skompilowany, zapis na nośniku).
Programy chronią też wystarczająco szczegółowe prace przygotowawcze i projektowe.
Ochrona to nie zależy od testów jakościowych ani estetycznych programu, ani od jego oryginalności.
Programy komputerowe są chronione w taki sam sposób co dzieła literackie.
Nie są chronione ich koncepcje i interfejsy (graficzne/tekstowe).
Pojedyncze elementy programów mogą być chronione z osobna.
Np. interfejs graficzny może być objęty ochroną utworu, ale osobno od ochrony samego programu komputerowego.

\subsubsection{Dekompilacja Programów Komputerowych (10.04)}

\paragraph{Interfejs}

logiczna lub fizyczna część systemu informatycznego, służąca komunikacji z jego innymi częściami i użytkownikami.

\paragraph{Interoperacyjność}

umożliwienie połączenia, współpracy i oddziaływania pomiędzy interfejsami.
Zdolność do wymiany informacji i wykorzystania informacji już wymienionych.

\paragraph{Dekompilacja}

tłumaczenie formy programu komputerowego do wersji wyższego poziomu, termin nie jest jednoznaczny z informatycznym. Jest dozwolona jeżeli służy osiągnięciu interoperacyjności danego programu z innymi.

\paragraph{Kto może dekompilować?}

Osoba będące licencjobiorcą, posiadające prawo do używania kopii programu, lub działająca w imieniu takich osób.

\paragraph{Kiedy można dekompilować?}

Gdy informacje konieczne do osiągnięcia interoperacyjności nie są łatwo dostępne i nie jest na celu naruszenie uzasadnionych interesów twórców programu ani wykorzystanie programu w celach innych niż standardowe.
Poprawienie błędu i~naprawy są uzasadnionym powodem.
Nie jest wymagane zezwolenie autora i nie wolno zabraniać napraw.


\section{Autorskie prawa zależne (20.03)}

\subsection{Utwory zależne}

Opracowania utworów pierwotnych takie jak tłumaczenia, przeróbki, adaptacje.
Można je tworzyć i rozpowszechniać za zgodą pierwotnego twórcy.
Wytworzenie ich jest bez uszczerbku dla~praw pierwotnych i pociąga za sobą prawa autora utworu zależnego.
Należy zaznaczyć, że~jest to utwór pierwotny i wymienić twórcę i tytuł utworu pierwotnego.
Aby wytworzyć utwór zależny należy uzyskać zgodę autora pierwotnego, o ile jego prawa nie wygasły.

\paragraph{Tłumaczenia}

nie zawsze noszą cechy utworów - tłumaczenia które mają w sobie mało inwencji,
jak np. tłumaczenia dokumentacji technicznej nie są utworami, w przeciwieństwie do~tłumaczeń dzieł literackich.

\paragraph{Utwory zależne a inspirowane}

Utwór inspirowany to taki, który zawiera pewne motywy zawarte w innym utworze.
Jeżeli motywy są zindywidualizowane do postaci zawartej w utworze pierwotnym, to nie jest to już utwór inspirowany.
Utwory inspirowane nie są utworami zależnymi

\section{Utwory niechronione prawem autorskim (10.04)}

Poniższe kategorie utworów są niechronione prawem autorskim.
Mogą natomiast być chronione jako symbole narodowe, znaki towarowe bądź przez inne akty prawne.

\begin{itemize}
  \item akty normatywne i ich projekty.
  \item urzędowe dokumenty, materiały, znaki i symbole
  \item opublikowane opisy patentowe, lub ochronne
  \item proste informacje prasowe (informacje zawierające same fakty, bez elementu twórczego)
\end{itemize}

Nie wszystkie materiały tworzone przez urzędy to materiały urzędowe.
Na przykład dokumenty wytworzone na potrzeby przetargów, jak najbardziej są chronione prawem autorskim.

\section{Podmiot prawa autorskiego (17.04)}

\subsection{Komu przysługują prawa autorskie?}

Prawa autorskie przysługują twórcy, o ile ustawa nie stanowi inaczej.
Domniemywa się, że~twórcą jest osoba, której autorstwo podano do publicznej wiadomości wraz z rozpowszechnianiem utworu.
Twórca nie musi ujawniać swojego autorstwa, wówczas w wykonywaniu prawa autorskiego zastępuje go wydawca, lub odpowiednia organizacja zbiorowo zarządzająca prawami autorskimi.

\subsection{Czy można odstąpić od praw autorskich?}

Ogólnie, nie można odstąpić praw innej osobie, jednak istnieją szczególne przypadki, na które pozwala ustawa.

\subsection{Utwory zbiorowe}

W przypadku utworu zbiorowego prawa autorskie do całości i tytułu przysługują wydawcy, natomiast do jego poszczególnych elementów, ich twórcom.

\subsubsection{Programy komputerowe}

Ogólnie są rozpatrywane jak utwory zbiorowe, jednak, jeśli powstają w wyniku wykonywania obowiązków wynikających ze stosunku pracy, prawa do nich przysługują pracodawcy, o ile umowa nie stanowi inaczej.

\subsubsection{Współtwórczość}

Współtwórcom przysługują wspólne prawa autorskie, wielkości udziałów są równe, o ile sąd nie orzecze inaczej na wniosek współtwórcy.
Każdy współtwórca może bez uszczerbku dla innych wykonywać prawa autorskie do swojej części utworu.
Do wykonania prawa autorskiego do całości utworu konieczna jest zgoda wszystkich współtwórców, w przypadku braku zgody rozstrzyga sąd.
Każdy ze współtwórców może dochodzić roszczeń w sprawie naruszenia praw do całości utworu.
Za współtwórcę uznaje się osobę, która włożyła w utwór działanie intelektualnej o twórczym charakterze, współtwórcy muszą działać w porozumieniu.
Pomysłodawca nie jest współtwórcą, nie można zostać współtwórcą wyłącznie na podstawie umowy.

\subsection{Utwory pracownicze}

Prawa majątkowe i zależne do utworów tworzonych na podstawie obowiązków wynikających ze stosunku pracy przechodzą na pracodawcę w chwili przyjęcia utworu (rozpoczęcia jego wykorzystywania), o ile UoP nie stanowi inaczej.
Utwory tworzone podczas konkursów, o ile ich regulamin nie stanowi inaczej, nie mają charakteru pracowniczego.
Nie można zawrzeć takiej umowy co do wszystkich utworów tworzonych w przyszłości przez twórcę.

\subsubsection{Programy komputerowe}

W przypadku programów komputerowych tworzonych na bazie stosunku pracy, prawa majątkowe do niego przysługują pracodawcy, co więcej, przysługują mu też prawa do integralności programu.

\section{Prawa autorskie osobiste i majątkowe}

\subsection{Prawa osobiste}

Chronią więź twórcy z utworem.
Szczególnie: prawo do autorstwa utworu i oznaczania go swoim nazwiskiem, lub pseudonimem.
Gwarantują nienaruszalność jego treści i formy, oraz prawo do nadzoru i decyzji o jego rozpowszechnianiu.
Ma osobisty charakter, więc jest niezbywalne. Jego działanie jest nieograniczone w czasie.

\subsubsection{Prawo do autorstwa utworu}

Twórcy przysługuje prawo do decydowania o tym jak zostanie oznaczony jego utwór.
Może użyć do tego własnego nazwiska lub wybranego pseudonimu i jakakolwiek zmiana tego oznaczenia bez jego zgody stanowi naruszenie tego prawa.

\subsubsection{Prawo do integralności utworu}

Twórcy przysługuje prawo do utrzymania integralności utworu, to znaczy do uchronienia go przed zmianami zrywającymi więź twórcy z utworem.
Prawo to gwarantuje również wyłączność decydowania o postaci w jakiej dzieło ma być rozpowszechniane.

\subsubsection{Prawo do decyzji o udostępnieniu utworu publiczności}

Prawo to daje twórcy prawa do decydowania czy dzieło ma zostać wydane, czy nie. Wygasa z chwilą pierwszego udostępnieniu utworu przez autora. Chroni również utwory przed rozpowszechnianiem wbrew wyraźnej woli zmarłego autora.

\subsubsection{Prawo do nadzoru nad sposobem użytkowania utworu}

Gwarantuje m. in. prawo do nadzoru nad sporządzaniem kopii utworów plastycznych, modyfikacjami utworów architektonicznych itp. na etapie pierwszego jak i kolejnych rozpowszechnień.

\subsubsection{Programy komputerowe}

Do programów komputerowych nie stosuje się praw do nienaruszalności formy i treści oraz nadzoru nad sposobem korzystania z utworu.

\subsubsection{Ghostwriting}

Dopuszczalny jest przy przemówieniach publicznych, natomiast w pozostałych przypadkach przywłaszczenie sobie lub wprowadzenie w błąd co do autorstwa jest przestępstwem.

W przypadku prac dyplomowych nosi to również znamiona korzystania z dokumentów poświadczających nieprawdę poprzez wprowadzenie w błąd funkcjonariusza publicznego.

\subsection{Prawa Majątkowe}

Dają twórcy wyłączne prawa do korzystania i rozporządzania utworem na wszystkich polach eksploatacji oraz do czerpania z niego korzyści majątkowych. Są zbywalne, podlegają zrzeczeniu i dziedziczeniu.

\subsubsection{Pola eksploatacji utworu}

Prawa do między innymi:
\begin{itemize}
  \item utrwalania i zwielokrotniania (powielania) utworu,
  \item wprowadzania do obrotu i obrotem egzemplarzami utworu, najmem oryginału lub egzemplarzy,
  \item publicznego rozpowszechniania, wykonywania, wyświetlania, nadawania i emitowania utworu.
\end{itemize}

\subsubsection{Licencjonowanie utworu}

Twórcy przysługuje prawo do zbycia, lub udzielenia praw do korzystania z konkretnych pól eksploatacji innym podmiotom.
Przyjmuje się, że licencja obejmuje wyłącznie wyraźnie w niej wymienione pola eksploatacji.

Jeżeli licencja nie wskazuje inaczej, twórcy przysługuje odrębne wynagrodzenie za korzystanie z każdego z pól eksploatacji. Nie można w umowie przenieść praw na wszystkich polach eksploatacji, ani zawrzeć jej w innej formie niż pisemna.

\subsubsection{Pisemna forma dokumentu}

Oświadczenie woli zawarte na piśmie, odręcznie podpisane przez stronę umowy. Do jego zawarcia wystarczy wymiana odręcznie podpisanych oświadczeń obu stron.

\paragraph{Forma elektroniczna}

oświadczenie woli w formie elektronicznej, podpisane kwalifikowanym podpisem elektronicznym. Zgodnie z rozporządzeniem eIDAS jest równoważna formie pisemnej.

\subsubsection{Zwielokrotnianie utworu}

Zakres pól eksploatacji, które powielają utwór w tej samej lub odmiennej formie. Każdy sposób i technika zwielokrotniania stanowi odmienne pole eksploatacji.

Przekaz utworu w sposób przejściowy, niemający znaczenia gospodarczego, a niezbędny w celu zgodnego z prawem korzystania z utworu, lub jego przekazu w systemie teleinformatycznym, nie stanowi zwielokrotniania.

\subsubsection{Rozpowszechnianie utworu}

Udostępnianie publiczne utworów za zgodą twórcy.

\paragraph{Reprodukcja trwała}

pozwala na wielokrotne korzystanie z utworu, jest to obrót oryginałem, lub egzemplarzami utworu.

\paragraph{Reprodukcja jednorazowa}

stanowi jednorazowe odtworzenie utworu odbiorcom, w określonym miejscu i czasie.

\subsubsection{Wyczerpanie praw autorskich}

Wprowadzenie oryginału lub egzemplarza utworu do obrotu na terenie Europejskiej Wspólnoty gospodarczej wyczerpuje prawo do zezwalania na jego dalszy obrót na terenie RP. Nie dotyczy to zezwalania na najem bądź użyczenie.
Co znaczy, że prawa do sprzedaży egzemplarzy już znajdujących się w obrocie nie przysługują twórcy.

\subsubsection{Sprzedaż, a licencjonowanie}

Produkty elektroniczne, takie jak ebooki, czy programy komputerowe często nie są sprzedawane w egzemplarzach, a rozpowszechniane na określonej licencji. Licencja (EULA) reguluje warunki korzystania i rozpowszechniania utworu. W ten sposób nie dochodzi do wyczerpania praw autorskich, gdyż nie jest wprowadzany do obrotu żaden egzemplarz.

\paragraph{Odsprzedanie licencji programu}

jest możliwe na takich samych zasadach jak odsprzedanie egzemplarza, z tym że muszą zostać spełnione konkretne warunki:
\begin{itemize}
  \item kopia programu musi być całkowicie przeniesiona,
  \item licencja nie może być ograniczona czasowo,
  \item licencja musi być opłacona w sposób pokrywający korzystanie z programu w przyszłości.
\end{itemize}
Odsprzedania licencji można dokonać tylko na terytorium EOG, nie można odwołać tego postanowieniami licencyjnymi. Licencje wolumenowe (OEM) również podlegają pod ten przepis.

\paragraph{Zmiana nośnika kopii programu komputerowego}

nie jest ty samym co odsprzedanie licencji, a zwielokrotnieniem go w innej formie. Nie jest dozwolona na podstawie legalności odsprzedania licencji, nawet jeżeli chodzi o legalnie sporządzoną kopię zapasową.

\subsubsection{Ebooki a programy komputerowe}

W przypadku ebooków, czy innych utworów w formie elektronicznej, niebędących programami komputerowymi, nie obowiązują przepisy zezwalające na odsprzedanie licencji.

\end{document}

% LocalWords:  Współtwórczość UoP Ghostwriting eIDAS wolumenowe
