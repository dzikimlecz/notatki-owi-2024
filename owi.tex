\documentclass{article}
\usepackage{graphicx}
\usepackage{transparent}
\usepackage[T1]{fontenc}
\usepackage[a4paper, total={15cm, 24cm}]{geometry}
\usepackage[polish]{babel}
\usepackage{amssymb}
\usepackage{amsmath}
\usepackage[
  pdfpagelabels=true,
  pdftitle={Notatki z OWI},
  pdfauthor={Jan Pulkowski},
  colorlinks=true,
  linkcolor=blue,
  ]{hyperref}
\usepackage[]{titlesec}
\usepackage[p]{scholax}

\newcommand{\watermark}{
  \begin{figure}[b]
    \transparent{0.4} \includegraphics*[]{mlecz}
    \centering
  \end{figure}
}

\setcounter{secnumdepth}{2}

\renewcommand{\thesection}{\arabic{section}}

\titleformat{\section}{\newpage\normalfont\Large\bfseries\scshape}{Wykład \thesection:}{1em}{}

\title{Notatki z Ochrony Własności Intelektualnej}
\author{mlecz}
\date{Semestr Letni 2024}

\begin{document}

\maketitle
\watermark

\tableofcontents
\watermark

\section{Prawo i jego źródła (28.02)}

\subsection{Prawo}

\paragraph{Prawo w ujęciu normatywnym}
to zespół norm i reguł określających postępowanie ludzi, ustanowionych, usankcjonowanych i zabezpieczanych przez aparat przymusu państwowego.

\paragraph{Prawo wg Kanta}
to reguły pozwalające na pogodzenie samowoli dwóch niezależnych jednostek.

\subsubsection{Zasady Prawa}
\begin{itemize}
  \item Lex posterior derogat legi priori -- prawo późniejsze uchyla prawo wcześniejsze
  \item Lex superiori derogat legi inferiori -- prawo o wyższej mocy uchyla prawo o niższej mocy
  \item Lex specialis derogat legi generali -- prawo szczegółowe uchyla prawo ogólne
  \item Ne bis in idem -- Nie można orzekać dwa razy w tej samej sprawie
  \item Lex retro non agit -- Prawo nie działa wstecz (o ile ustawa późniejsza nie jest
        korzystniejsza dla oskarżonego)
\end{itemize}

\subsubsection{Miejsca zapisu prawa}
\begin{itemize}
  \item \textbf{Dziennik ustaw -- jedyne oficjalne miejsce zapisu prawa}
  \item Monitor Polski -- służy do ogłaszania wewnętrznych aktów prawnych wydawanych
        przez organy państwowe
  \item ISAP -- Internetowy System Aktów Prawnych
  \item Systemy Prawne (LEX, Legalis) -- ujednolicony i czytelny zapis informacji
        z dziennika ustaw z odnośnikami do innych powiązanych informacji prawnych
\end{itemize}

\subsection{Język prawny a prawniczy}

\paragraph{Język prawny} to język tworzenia prawa, w którym tworzone są akty prawne.
Nie musi być poprawny w sensie składni j. pol. ale jest mniej wieloznaczny i ściślejszy
od języka naturalnego.

\paragraph{Język prawniczy} to język stosowania i doktryny prawa, omawiający treść
i interpetację ustawy.

\paragraph{Język prawny a matematyczny}
Język prawny często celowo zostawia pewne pojęcia niedookreślone,aby pozostawić
sądom pole do odmiennej wykładni prawa w różnych przypadkach.

\subsection{Obowiązywanie Prawa}

\paragraph{Prawo powszechnie obowiązujące} to przepisy adresowane do wszystkich podmiotów.
Wyznacza ich sytuację prawną, tzn. prawa i obowiązki.
Stanowi fundament działania państwa,
jest umową społeczną między jego państwa i obywatelami.
Państwo może wykonywać jedynie czynności \textit{explicite} dozwolone przez prawo,
a obywatele wszystkie czynności niezakazane.

\subsection{Źródła prawa}
\begin{enumerate}
  \item Konstytucja
  \item Ustawy
  \item Ratyfikowane umowy międzynartodowe
  \item Rozporządzenia
  \item Akty prawa miejscowego (tylko na obszarze obowiązywania)
\end{enumerate}

\paragraph{Konstytucja}-- akt prawny nadrzędny wobec wszystkich pozostałych.
Jej przepisy stosuje się bezpośrednio, o ile sama nie stanowi inaczej.
Jest jednym, spójnym aktem prawnym. Konstytucja jest celowo napisana w sposób ogólny, aby mogły
uszczegółowić ją ustawy.
Określa bazowe ramy prawne, sposób stanowienia prawa i zasady jakie można z niej wyprowadzać.

\paragraph{Ustawa, a rozporządzenie} -- Ustawa to prawo stanowione przez władzę ustawodawczą,
rozporządzenie jest aktem wykonawczym, doprecyzowuje działanie i stosowanie ustawy oraz określa
szczegółowe procedury jej realizacji. Roporządzenia są tworzone na podstawie ustaw.

\subsubsection*{\textit{koniec wykładu 28.02}}

\subsection{Prawo unijne}

\paragraph{Prawo pierwotne}
to umowy międzynarodowe tworzące i kształtujące Unię Europejską. Stanowią jej swoistą konstytucję.
Na przykład Traktat z Lizbony powołujący Unię.

\paragraph{Prawo wtórne}
to rozporządzenia, dyrektywy, decyzje, opinie i zalecenia wydawane przez organy Unii.

\paragraph{Rozporządzenia unijne} to akty prawne
bezpośrednio i jednolicie działające we wszystkich krajach członkowskich.

\paragraph{Dyrektywy unijne} to akty prawne,
wskazujące cele i wytyczne, które ustawodawstwo poszczególnych krajów członkoskich powinno zrealizować.
Muszą zostać wydane krajowe akty prawne, które implementują dyrektywę.

\subsubsection{Prawo polskie a prawo unijne}
Konstytucja dopuszcza przekazanie części uprawnień organów państwowych organom prawa międzynarodowego.
Proeuropejska wykładnia prawa jest ponad krajową --
ratyfikowana umowa międzynarodowa ma pierwszeństwo wobec sprzecznej z nią ustawy.
Nad umowami międzynarodowymi stoi jedynie konstytucja.


\section{Własność Intelektualna (06.03)}

\subsection{Dobra Niematerialne}

\paragraph{Dobro niematerialne (intelektualne)}
to wytwór myśli, cechujący się twórczym charakterem, istniejący w świadomości człowieka
i mogący podlegać ochronie prawnej niezależnie od swojego ewentualnego nośnika \textit{(Corpus Mechanicum)}.

\subsubsection{Rodzaje dóbr niematerialnych}
\begin{itemize}
  \item Utwory (w rozumieniu prawa autorskiego)
  \item Rozwiązania -- zaplanowane koncepty, prowadzące do realizacji danego celu (wynalazki, projekty, wzory)
  \item Oznaczenia, symbole (znaki towarowe, oznaczenia geograficzne)
\end{itemize}

Dobra niematerialne mogą obejmować więcej niż jeden z powyższych aspektów.

\subsection{Prawo własności intelektualnej}

\paragraph{Prawo autorskie i pokrewne}
zajmuje się utworami. Godzi prawa twórców z możliwością rozwoju społeczeństwa. Chroni prawa twórców jak i wykonawców. Regulowane międzynarodowo.

\subsection{Konwencje o ochronie prawa autorskiego}

\subsubsection{Konwencja Berneńska}

\paragraph{Konwencja Berneńska}
o ochronie dzieł literackich i artystycznych z 1886 roku. Powstała z inicjatywy m. in. Wiktora Hugo.
Wprowadziła koncept międzynarodowej ochrony praw autora.
Ratyfikowana przez większość państw świata, wielokrotnie nowelizowana. Nadal obowiązuje.

\paragraph{Zasada wzajemnego respektowania praw} Utworom zagranicznym zapewnia się taką samą ochronę jak utworom krajowym.

\paragraph{Zasada automatyzmu}
Ochrona praw autorskich przysługuje każdemu utworowi od momentu jego powstania.

\subsubsection{Inne Konwencje}

\paragraph{Konwencja genewska} o prawach autorskich.
Zapewnia alternatywę wobec Konwencji Berneńskiej.
Reprezentuje bardzie liberalne podejście do sprawy.

\paragraph{Konwencja Rzymska}
o ochronie wykonawców, producentów fonogramów oraz organizacji nadawczych.
Zapewnia ochronę twórcom utworów dźwiękowych.

\paragraph{Porozumienie w sprawie Handlowych Aspektów Praw Własności Intelektualnej (TRIPS)}
poza zakresem Konwencji Berneńskiej zapewniło ochronę programów komputerowych i baz danych.
Chroni programy w postaci kodu źródłowego i przedmiotowego (postać binarna, wynikowa, ang. \textit{object files}).

\paragraph{Traktaty Światowej Organizacji Własności Intelektualnej (WIPO)}
o prawie autorskim, o artystycznych wykonaniach i fonogramach.
Pierwszy traktat międzynarodowy nakładający obowiązek chronienia programów komputerowych i baz danych.
Wymusza wprowadzenie sankcji za łamanie i obchodzenie zabezpieczeń programów.

\subsection{Ustawa o prawach autorskich i pokrewnych}

\subsubsection{Utwory}

\paragraph{Utwór} to każdy przejaw działalności twórczej o indywidualnym charakterze, ustalony w jakielkolwiek postaci,
niezależnie od wartości, przeznaczenia i sposobu wyrażenia.
Definicja celowo jest szeroka aby móc pozostawić kwestię, czy dana rzecz jest utworem do interpretacji sądowej.

\paragraph{Konkretne przypadki wykładni definicji utworu}
\begin{itemize}
  \item Smak i zapach nie może podlegać ochronie prawnoautorskiej. Subiektywne doznania odgrywają tu zbyt dużą rolę.
  \item By mówić o utworze konieczna jest cecha nowości, twórca musiał wprowadzić uchwytne elementy, nieobecne w dotychczasowym stanie rzeczy.
  \item Jeżeli wysoce prawdopodobne jest, że ktoś w przyszłości wytworzy \textbf{identyczny} przedmiot, to nie mówi się o utworze.
\end{itemize}

\paragraph{Zasada Ustaloności}
-- Utwór musi zostać przedstawiony, w postaci która pozwala na jego percepcję komuś innemu niż twórcy, nawet ulotnej, nietrwałej.
Utwór może istnieć bez nośnika, a posiadanie nośnika nie jest związane z prawem do utworu.




\end{document}
