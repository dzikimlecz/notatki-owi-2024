\documentclass{article}
\usepackage{graphicx} % Required for inserting images
\usepackage[T1]{fontenc}
\usepackage[a4paper, total={15cm, 24cm}]{geometry}
\usepackage[polish]{babel}
\usepackage{amssymb}
\usepackage{amsmath}
\usepackage[
  pdfpagelabels=true,
  pdftitle={Notatki z OWI},
  pdfauthor={Jan Pulkowski},
  colorlinks=true,
  linkcolor=blue,
  ]{hyperref}
\usepackage[]{titlesec}
\usepackage[p]{scholax}

\setcounter{secnumdepth}{2}

\renewcommand{\thesection}{\arabic{section}}

\titleformat{\section}{\normalfont\Large\bfseries\scshape}{Wykład \thesection:}{1em}{}

\title{Notatki z Ochrony Własności Intelektualnej}
\author{mlecz}
\date{Semestr Letni 2024}


\begin{document}
\maketitle

\tableofcontents

\section{Prawo i jego źródła (28.02)}

\subsection{Prawo}

\paragraph{Prawo w ujęciu normatywnym}
to zespół norm i reguł określających postępowanie ludzi, ustanowionych, usankcjonowanych i zabezpieczanych przez aparat przymusu państwowego.

\paragraph{Prawo wg Kanta}
to reguły pozwalające na pogodzenie samowoli dwóch niezależnych jednostek.

\subsection{Zasady Prawa}
\begin{itemize}
  \item Lex posterior derogat legi priori -- prawo późniejsze uchyla prawo wcześniejsze
  \item Lex superiori derogat legi inferiori -- prawo o wyższej mocy uchyla prawo o niższej mocy
  \item Lex specialis derogat legi generali -- prawo szczegółowe uchyla prawo ogólne
  \item Ne bis in idem -- Nie można orzekać dwa razy w tej samej sprawie
  \item Lex retro non agit -- Prawo nie działa wstecz (o ile ustawa późniejsza nie jest
        korzystniejsza dla oskarżonego)
\end{itemize}

\subsection{Miejsca zapisu prawa}
\begin{itemize}
  \item \textbf{Dziennik ustaw -- jedyne oficjalne miejsce zapisu prawa}
  \item Monitor Polski -- służy do ogłaszania wewnętrznych aktów prawnych wydawanych
        przez organy państwowe
  \item ISAP -- Internetowy System Aktów Prawnych
  \item Systemy Prawne (LEX, Legalis) -- ujednolicony i czytelny zapis informacji
        z dziennika ustaw z odnośnikami do innych powiązanych informacji prawnych
\end{itemize}

\subsection{Język prawny a prawniczy}

\paragraph{Język prawny} to język tworzenia prawa, w którym tworzone są akty prawne.
Nie musi być poprawny w sensie składni j. pol. ale jest mniej wieloznaczny i ściślejszy
od języka naturalnego.

\paragraph{Język prawniczy} to język stosowania i doktryny prawa, omawiający treść
i interpetację ustawy.

\paragraph{Język prawny a matematyczny}
Język prawny często celowo zostawia pewne pojęcia niedookreślone,aby pozostawić
sądom pole do odmiennej wykładni prawa w różnych przypadkach.

\subsection{Obowiązywanie Prawa}

\paragraph{Prawo powszechnie obowiązujące} to przepisy adresowane do wszystkich podmiotów.
Wyznacza ich sytuację prawną, tzn. prawa i obowiązki.
Stanowi fundament działania państwa,
jest umową społeczną między jego państwa i obywatelami.
Państwo może wykonywać jedynie czynności \textit{explicite} dozwolone przez prawo,
a obywatele wszystkie czynności niezakazane.

\subsection{Źródła prawa}
\begin{enumerate}
  \item Konstytucja
  \item Ustawy
  \item Ratyfikowane umowy międzynartodowe
  \item Rozporządzenia
  \item Akty prawa miejscowego (tylko na obszarze obowiązywania)
\end{enumerate}

\paragraph{Konstytucja}-- akt prawny nadrzędny wobec wszystkich pozostałych.
Jej przepisy stosuje się bezpośrednio, o ile sama nie stanowi inaczej.
Jest jednym, spójnym aktem prawnym. Konstytucja jest celowo napisana w sposób ogólny, aby mogły
uszczegółowić ją ustawy.
Określa bazowe ramy prawne, sposób stanowienia prawa i zasady jakie można z niej wyprowadzać.

\paragraph{Ustawa, a rozporządzenie} -- Ustawa to prawo stanowione przez władzę ustawodawczą,
rozporządzenie jest aktem wykonawczym, doprecyzowuje działanie i stosowanie ustawy oraz określa
szczegółowe procedury jej realizacji. Roporządzenia są tworzone na podstawie ustaw.

\subsubsection*{\textit{koniec wykładu 28.02}}




\end{document}
